\documentclass{article}
\usepackage[utf8]{inputenc}
\usepackage[margin=1.00in]{geometry}
\usepackage{amssymb,amsmath}
\setlength{\parskip}{.75em}

\begin{document}

\title{%
    Python Data Analysis Guide \\
}
\author{Ayman Ali}
\date{}
\maketitle

\clearpage

% Table of Contents
\tableofcontents
\listoffigures
\listoftables

\clearpage

\section{Dataframe Manipulation with Pandas}
\begin{enumerate}
    \item Create dataframes based on value in column:
    \begin{verbatim}
new_df = df.loc[df['Column'] == value]
    \end{verbatim}
    \item Drop columns:
    \begin{verbatim}
drop_columns = ['Column 1', 'Column 2', 'Column 3']
new_df = df.drop(drop_columns, axis = 1)
    \end{verbatim}
    \item Select rows by index (works with slicing as well):
    \begin{verbatim}
rows = df.iloc[index]
    \end{verbatim}
    \item Select row and column by index (can use slicing on both):
    \begin{verbatim}
new_df = df.iloc[index_of_row, index_of_column]
    \end{verbatim}
    \item Replace values:
    \begin{verbatim}
new_df = df.replace(to_replace = value_to_replace,
                    value = value_to_replace_with)
    \end{verbatim}
\end{enumerate}

\section{Graphing with Matplotlib}
\subsection{General Setup}
\begin{enumerate}
    \item For the imports:
    \begin{verbatim}
import matplotlib.pyplot as plt
# if you want to see what styles you can pick from
plt.style.available
# personal preference
plt.style.use('bmh')
    \end{verbatim}
    \item Set up a graph that you're about to plot with:
    \begin{verbatim}
'''
If both columns and rows are greater than 1, then ax will be a (row, column)
dimensional array. Otherwise, it'll be a one-dimensional array, with either
(rows) or (columns) depending on which one is greater than 1. You can access
each axis with standard indexing (i.e. ax[1][2] will give you the axes of
the graph in row 2 and column 3)
''' 
fig, ax = plt.subplots(nrows = num_rows, ncols = num_cols,
                       sharex = boolean, sharey = boolean, dpi = integer,
                       figsize = (width, height))
    \end{verbatim}
    \item Titles:
    \begin{verbatim}
# Individual axis titles with
ax.set_title('string')
# For a general graph title, use text boxes (next section)
    \end{verbatim}
    \item Adding text boxes to the figure:
    \begin{verbatim}
# Find all kwargs at the following URL:
# https://matplotlib.org/api/text_api.html#matplotlib.text.Text
# Some of the more useful kwargs that I use are:
# verticalalignment ('center', 'top', 'bottom', 'baseline') and
# horizontalalignment ('center', 'right', 'left')
fig.text(double_xloc, double_yloc, 'text', **kwargs)
    \end{verbatim}
    \item Finishing the plot:
    \begin{verbatim}
# If you had a label anywhere then make sure to call it
plt.legend(fontsize = integer or None)
plt.savefig(file_path)
plt.close()
plt.clf()
    \end{verbatim}
\end{enumerate}
\subsection{Plot Types}
\begin{enumerate}
    \item Standard plot:
    \begin{verbatim}
# fill in 
    \end{verbatim}
    \item Scatter:
    \begin{verbatim}
# fill in 
    \end{verbatim}
    \item Bar graphs:
    \begin{verbatim}
spacing = double_val
plt.bar(x_values +/-/nothing spacing, y_vals, width = spacing,
        align = 'center', color = any matplotlib color)
    \end{verbatim}
    \item Errorbars:
    \begin{verbatim}
axis.errorbar(x_vals, y_vals, xerr = error_x, yerr = error_y,
              markersize = integer,
              lolims / uplims / xlolims / xuplims = optional booleans)
    \end{verbatim}
\end{enumerate}

\subsection{Misc.}
\begin{enumerate}
    \item Custom tick marks and steps:
    \begin{verbatim}
plt.xticks(np.arange(start = start_val, stop = stop_val, step = step_val))
plt.yticks(np.arange(start = start_val, stop = stop_val, step = step_val))
    \end{verbatim}
\end{enumerate}


\end{document}
